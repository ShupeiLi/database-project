\documentclass[12pt]{article}

\usepackage{geometry}
\usepackage[hidelinks, bookmarks = true]{hyperref}
\usepackage[numbered]{bookmark}
\usepackage{titling}
\usepackage{titlesec}
\usepackage{fontspec}
\usepackage{xeCJK}
\usepackage{setspace}
\usepackage{booktabs}
\usepackage{float}
\usepackage{graphicx}
\usepackage{geometry}
\usepackage{titletoc}
\usepackage{indentfirst}
\usepackage{fancyhdr} 
\usepackage{longtable}
\usepackage{supertabular}
\usepackage[normalem]{ulem}
\usepackage{listings}
\usepackage{xcolor}
\usepackage{xurl}
\usepackage{tikz}

\def\checkmark{\tikz\fill[scale=0.4](0,.35) -- (.25,0) -- (1,.7) -- (.25,.15) -- cycle;} 

\definecolor{codegreen}{rgb}{0,0.6,0}
\definecolor{codegray}{rgb}{0.5,0.5,0.5}
\definecolor{codepurple}{rgb}{0.58,0,0.82}
\definecolor{backcolour}{rgb}{0.95,0.95,0.92}

\lstdefinestyle{mystyle}{
	backgroundcolor=\color{backcolour},   
	commentstyle=\color{codegreen},
	keywordstyle=\color{magenta},
	numberstyle=\tiny\color{codegray},
	stringstyle=\color{codepurple},
	basicstyle=\ttfamily\footnotesize,
	breakatwhitespace=false,         
	breaklines=true,                 
	captionpos=b,                    
	keepspaces=true,                 
	numbers=left,                    
	numbersep=5pt,                  
	showspaces=false,                
	showstringspaces=false,
	showtabs=false,                  
	tabsize=2
}

\lstset{style=mystyle}
\pagestyle{fancy}

\fancyhead[R]{\thepage}
\fancyhead[L]{\leftmark}
\fancyfoot{}
\fancyfoot{}

\setmainfont{Source Serif Pro}
\setCJKmainfont[BoldFont=SimHei]{SimSun}
\setCJKmonofont{SimSun}
\geometry{a4paper, left = 2.9cm, right = 3cm, top = 2cm, bottom = 2.3cm, includehead}
\setlength{\parindent}{2em} 

\renewcommand{\baselinestretch}{1.5}
\renewcommand\tablename{表}
\renewcommand{\contentsname}{
		目\phantom{空格}录
}

\titlecontents{section}[2.3em]
{}
{\bfseries\contentslabel[\thecontentslabel]{1.5em}\MakeUppercase}
{\hspace*{-2.3em}\bfseries\MakeUppercase}
{\titlerule*[1pc]{.}\contentspage}

\titlecontents{subsection}[4.6em]
{}
{\bfseries\contentslabel{2em}}
{\hspace*{-2.3em}\bfseries}
{\titlerule*[1pc]{.}\contentspage}

\titlecontents{subsubsection}[6.9em]
{}
{\bfseries\contentslabel{2.5em}\itshape\space}
{\hspace*{-2.3em}\bfseries}
{\titlerule*[1pc]{.}\contentspage}

\makeatletter
\renewcommand\tableofcontents{%
	\section*{\centerline{\MakeUppercase{\contentsname}}
		\@mkboth
		{\MakeUppercase\contentsname}
		{\MakeUppercase\contentsname}
	}%
	\@starttoc{toc}%
}
\makeatother

\begin{document}
% cover
\thispagestyle{empty}
\begin{center}
	\includegraphics[width = 13cm]{fudan-title.png}
\end{center}
\vspace{0.2cm}
\begin{center}
	\includegraphics[width = 4cm]{fudan-logo.jpg}
\end{center}
\vspace{0.2cm}
\begin{center}
\Huge \textbf{小组项目报告}
\end{center}
\begin{center}
\Huge \textbf{平安物流:疫情防控下的物流监测系统}
\end{center}
\vspace{1cm}
\begin{center}
\begin{minipage}{0.6\linewidth}
\begin{flushleft}
\Large 课程名称:数据库与企业数据管理 \\
指导老师:张成洪\phantom{空}教授 \\
\vspace{0.5cm}
\phantom{空格空格空格}第三组 \\
\phantom{空}丁语欣\phantom{空格}高\phantom{空}畅\phantom{空格}李舒沛\\  
\phantom{空}张高阳\phantom{空格}张秋岑\\
\phantom{空格空}(按姓名首字母排序)
\end{flushleft}
\end{minipage}
\end{center}

% contents
\newpage
\thispagestyle{empty}
\tableofcontents

% Text
\newpage
\section{选题}

\subsection{选题背景}

随着疫情常态化,政府部门、医疗系统、各级单位及普通群众已经形成了较为成熟的疫情防控系统。然而,虽然对“人传人”的新冠疫情防护和响应措施已经非常完善,然而最近出现了对新冠病毒“物传人”的担忧。北京市海淀区一居民购自内蒙古的物品,外包装被检测出阳性结果,证实了存在“物传人”的风险。11月10日,中国疾控中心专门针对外卖配送和快递从业人员发布了健康防护指南;中国疾控中心消毒学首席专家张流波在11月13日的国务院联防联控机制新闻发布会上建议,在收快递时做好适度防护,表面包装要求不带入室内或进行消毒,处理完快递后做好手卫生。\par 
在实际应用中我们发现,在收快递时做好防护实施起来难度较大。一方面,人的跨区域流动减速,但电商和快递却不能停滞,需要物件的流动供给人们日常生活的需要;另一方面,物流运输网络比较复杂,数量大、链条长、涉及的人员和区域多,使得对疫情的追踪管控变得非常困难。\par 
面对疫情局势的频繁变化,与复杂的物流网络,用户显得有些手足无措。虽然目前已经有快递路线跟踪和快递人员健康管理系统,但颗粒度较粗、反应不够敏捷、查询较复杂,难以满足当下严格的防护需求。比如在菜鸟平台,寄件人和收件人可以查询快递途径的转运点信息及“最后一位快递员”的体温信息,但无法直观获取疫情与快递的关联信息,需要用户自行拼接电商平台提供的地理位置信息和疫情风险地区信息。\par 
在这样的背景下,我们认为有必要搭建起电商、物流与疫情信息的桥梁,将多方信息整合起来,为多方提供一站式查询服务和解决方案。

\subsection{选题意义}

本题将搭建疫情防控下的物流监测系统,能够一站式查询潜在的物流相关疫情风险。除此之外,基于疫情防控需求搭建起的中台能力(不同来源数据之间的关联、数据统计和可视化)可以应用到更多疫情之外的分析场景。

\subsubsection{对电商平台}

这一信息系统的搭建,不仅使电商平台更方便直观地监控订单情况,还为提供疫情相关增值服务奠定基础。在信息汇总时,电商平台可以将订单、物流信息简单地通过第三方平台关联起来,而不用内部开发,且可以利用可视化和统计工具作更深入的收益、订单分析;此外,还可以提供快递信息查询的增值服务提升竞争力,也可以创新如疫情险的附加产品。

\subsubsection{对物流公司}

物流公司能够通过接入系统,更加轻松地管理快递员与货物。尤其是在突发事件发生时(如某区域出现病例),物流公司能够快速识别到所有可能被影响的件与人,进行针对性隔离或检查。

\subsubsection{对买家}

买家可以查询到更精细、全面的物流信息,做到心中有数,物流相关防疫信息全知晓。当收件人存在潜在风险时,可以在本系统的帮助下迅速识别风险,自我防护。

\subsubsection{对卖家}

当部分地区存在疫情风险,可能出现快递停运的情况。系统可以将收货地的疫情信息展示给卖家,起到提醒卖家尽早取消订单、避免不必要的损失的作用。

\subsection{范围界定}

本系统作为第三方平台,为物流公司和电商平台提供基于多方信息给出数据或数据分析结论的中间服务。系统将整合物流订单信息、承运快递人员和买方卖方信息、各地区疫情信息等,实现查询、分析、可视化相关数据的功能。

\subsubsection{用户}

\begin{itemize}
	\item 电商平台
	\item 物流公司
	\item 卖家
	\item 买家
\end{itemize}

\subsubsection{功能}

\noindent \textbf{电商平台}
\begin{enumerate}
	\item 输入买方用户信息、卖方用户信息、商品订单信息,支持手工输入、文件导入、MySQL数据库接入。
	\item 更改及删除买方用户信息、卖方用户信息、商品订单信息,支持单条记录操作及批量操作。
	\item 管理电商平台提供的数据访问权限,管理电商账户信息。
	\item 查询商品订单疫情相关信息。
\end{enumerate}

\noindent \textbf{物流公司}
\begin{enumerate}
	\item 输入员工健康信息、物流人员信息、物流订单信息、物流位置信息,支持手工输入、文件导入、MySQL数据库接入。
	\item 更改及删除员工健康信息、物流人员信息、物流订单信息、物流位置信息,支持单条记录操作及批量操作。
	\item 管理物流公司提供的数据访问权限,管理公司账户信息。
	\item 查询物流运输疫情相关信息。
\end{enumerate}

\noindent \textbf{卖家}
\begin{enumerate}
	\item 管理账户数据访问权限,管理账户信息。
	\item 查询卖家订单物流的疫情相关信息。
\end{enumerate}

\noindent \textbf{买家}
\begin{enumerate}
	\item 管理账户数据访问权限,管理账户信息。
	\item 查询购买商品物流配送的疫情相关信息。
\end{enumerate}
\noindent \textbf{数据统计与可视化:}创建针对不同类型用户生成相应视图,并进行统计分析及数据可视化。

\vspace{0.2cm}
\noindent \textbf{疫情情况信息实时更新:}定时爬虫抓取及储存最新疫情信息。

\section{需求分析}

\subsection{需求分析的过程描述}

为了深入了解信息系统各方用户的需求,我们小组采用了专人采访和问卷发放相结合的方式进行调研。我们的采访对象是数据库设计目标的多方服务对象,即电商平台、物流公司、商品买卖双方。我们希望能够通过实际调研了解到疫情防控下的物流风险给各方用户带来的难处,也希望能够更准确地了解各目标用户的真实需求并获得数据库设计上的建议。

\subsubsection{专人采访:对电商平台运营人员的采访}

某电商平台运营人员A表示,她在平日的工作中需要对商品订单信息、买卖双方的用户信息等实行监测与日常维护。此外,若遇到其他部门同事需要某些特定数据,她需要配合工作协助查询相关数据并将信息提供给同事。在疫情期间,整个电商平台往往需要多方数据来实现订单评估、业务反馈、业务预警、地区风险提示等动作,才能有效降低订单积压的可能性,提升平台的总体业务效率。

\subsubsection{专人采访:对物流公司快递员的采访}

顺丰快递小哥B表示自疫情防控以来,公司对包裹收发、车辆管理、员工身体状况监控等情况实行了更为严格的管理。他们每天都要测量并登记体温,而且每一次快递配送记录都需要准确地记录下来,以备突发疫情时的流调检查。他说公司会根据物流订单信息和位置信息进行提示,也会对快递包裹会实行消毒等操作,而他自己也可以查看到分派给自己的货物途径的地区。他表示如果能更准确地了解快递的物流信息,他也能及时地提醒收件人做好消毒工作,提高防疫意识。

\subsubsection{专人采访:对商品卖家的采访}

淘宝某日用品店家C表示,他曾遇到过多次买家地址处于中高风险区或者快递途径中高风险区的情况。因此他也会特别关注物流的最新信息,从而可以尽早提醒卖家来避免订单积压或者买家的埋怨和差评。此外,在疫情反复的时期,他都会选择和较大的物流公司合作。这些物流公司会采取更规范更严格的防控措施,这让他对商品在快递过程中的安全性更放心。

\subsubsection{对商品买家的问卷调研}

针对商品买家在疫情防控情况下对物流风险的关注情况,我们小组制作了一份调研问卷,通过朋友圈、微信群等途径进行分发,覆盖各年龄段人群,共计回收98份有效问卷。

\noindent \textbf{基础信息}

\noindent \textbf{Q1.} 您的年龄是多少?
\begin{center}
	\includegraphics[height=6cm]{survey1.png}
\end{center}

\noindent \textbf{Q2.} 您平均每周网购的频率是多少?
\begin{center}
	\includegraphics[height=6cm]{survey2.png}
\end{center}

\noindent \textbf{主体问题}

\noindent \textbf{Q3.} 您在网购时是否会关注物流信息?
\begin{center}
	\includegraphics[height=6cm]{survey3.png}
\end{center}

\noindent \textbf{Q4.} 您对目前物流信息详细程度的呈现是否满意?
\begin{center}
	\includegraphics[height=6cm]{survey4.png}
\end{center}

\noindent \textbf{Q5.} 疫情期间,您在多大程度上相信到手的快递包裹是安全的?
\begin{center}
	\includegraphics[height=6cm]{survey5.png}
\end{center}

\noindent \textbf{Q6.} 疫情期间,您会主动关注物流快递途径地区包含中高风险地区吗?
\begin{center}
	\includegraphics[height=6cm]{survey6.png}
\end{center}

\noindent \textbf{Q7.} 疫情期间,您是否会对取回的快递包裹进行消毒等防疫措施?
\begin{center}
	\includegraphics[height=6cm]{survey7.png}
\end{center}

\noindent \textbf{结论}

通过分析问卷结果,我们发现大约54\%的调查对象会在网购后较为经常地关注物流信息,而大约93\%的调查对象都对目前物流信息详细程度表示出满意态度。疫情防控期间,89\%的调查对象表示相信他们到手的快递是安全的,有35\%的调研对象会经常或总是关注快递途径地区是否包含中高风险地区。针对取回的快递,只有7\%的调研对象会经常或总是对包裹进行消毒措施。此外,有调研对象在问卷中表示,他们希望能够获得更直观清晰或更具警示性的提示来告诉他们可能存在的物流风险,这样他们在取快递时就会多一些防疫意识。

因此,我们希望我们建立的物流风险监测平台不仅能够更好地满足各服务对象对疫情防控期间物流管理的需求,也能通过该平台提升各方对物流风险的重视程度,从而规范自身操作,提升防疫意识,让每一份快递都能平安地送到消费者手中。

\newpage
\subsection{组织机构、用户分类及业务流程}

\subsubsection{组织机构}

\noindent \textbf{电商平台组织机构}
\begin{center}
	\includegraphics[height=7.8cm]{e-cormmerce_organization.png}
\end{center}

\noindent \textbf{物流公司组织机构}
\begin{center}
	\includegraphics[height=11.9cm]{logistics_organization.png}
\end{center}

\subsubsection{用户分类}

\noindent \textbf{物流公司}
\begin{itemize}
	\item 输入:物流管理、配送人员每日健康状况以及配送信息。
	\item 输入:运输包裹信息以及包裹配送所经地。
	\item 查询:特定包裹的配送状态与配送人员、产品相关信息。
	\item 查看特定时间段内包裹相关信息分析报告。
	\item 查看特定时间段内物流成本收益分析报告。
	\item 实现疫情下物流配送人员及包裹管理:对标国家卫生健康委员会官方网站,即使标注途径标为中高风险地区的人员及包裹。
\end{itemize}

\noindent \textbf{电商平台}
\begin{itemize}
	\item 输入:订单信息(包括买家、卖家信息)。
	\item 查询:卖家基本信息及相关售卖商品信息及配送状况。
	\item 查询:买家基本信息及相关购买商品信息及配送状况。
	\item 查询:特定包裹的配送状态与配送人员、产品相关信息。
	\item 查看电商平台成本收益分析报告。
	\item 查看特定时间段内电商平台总体订单配送信息分析报告。
\end{itemize}

\noindent \textbf{卖家}
\begin{itemize}
	\item 输入:卖家基本信息及对应相关电商平台的特定序列号。
	\item 查询:相关售卖商品信息及配送状况。
\end{itemize}

\noindent \textbf{买家}
\begin{itemize}
	\item 输入:买家基本信息。
	\item 查询:相关购买商品信息及配送状况。
\end{itemize}

\noindent \textbf{游客(进入网站未登录)}
\begin{itemize}
	\item 查看平台主业务。
	\item 对标国家卫生健康委员会官方网站,查看中高风险地区总地图及相关信息。
\end{itemize}

\subsubsection{业务流程}

\noindent \textbf{(1) 物流订单数据管理服务}
\begin{center}
	\includegraphics[height = 7cm]{logistics_order_mag.png}
\end{center}

\noindent \textbf{(2) 查询特定信息服务}
\begin{center}
	\includegraphics[height = 6.2cm]{query_info.png}
\end{center}

\noindent \textbf{(3) 查看特定分析报告服务}
\begin{center}
	\includegraphics[height = 6cm]{query_report.png}
\end{center}

\noindent \textbf{(4) 查看中高风险地区地图及相关信息服务}
\begin{center}
	\includegraphics[height = 3.5cm]{cov_map.png}
\end{center}

\noindent \textbf{(5) 数据统计与可视化流程}
\begin{center}
	\includegraphics[height = 6.7cm]{data_visualization.png}
\end{center}

\newpage
\subsection{数据流图}

\subsubsection{第一层数据流图}

\begin{center}
	\includegraphics[height = 20.5cm]{level-1.png}
\end{center}

\subsubsection{第二层数据流图}

\noindent \textbf{(1) 提交物流订单}
\begin{center}
	\includegraphics[height = 8cm]{level-2.1.png}
\end{center}

\noindent \textbf{(2) 安排配送}
\begin{center}
	\includegraphics[height = 12.5cm]{level-2.2.png}
\end{center}

\noindent \textbf{(3) 识别物流风险}
\begin{center}
	\includegraphics[height = 8cm]{level-2.3.png}
\end{center}

\subsection{数据字典}

\subsubsection{数据项}

\tablehead{
	\toprule
	\textbf{编号} & \textbf{数据项名} & \textbf{含义说明} & \textbf{数据类型} & \textbf{长度} & \textbf{备注} \\
	\midrule}
\tabletail{\bottomrule}

\begin{center}
\begin{supertabular}{llllll}
	1 &	Uname &	买方名称 &	Char &	30 &	非空 \\
	2 &	Uno &	买方编号 &	Integer & 4 &	非空 ,买方用户信  \\ &&&&& 息表主码\\
	3 &	Uaddress &	买方地址 &	Char &	30 &	非空 \\
	4 &	Uphone &	买方联系方式 &	Integer &	4 &	非空 \\
	5 &	Sname &	卖方名称 &	Char &	30 	& 非空\\
	6 &	Sno &	卖方编号 &	Integer &	4 	& 非空,卖方用户信 \\ &&&&& 息表主码 \\
	7 &	Saddress &	卖方仓储位置 &	Char &	30 	& 非空 \\
	8 &	Pname &	配送人员姓名 &	Char &	30 &	非空 \\
	9 &	Pno &	配送人员编号 &	Integer &	4 &	非空 \\
	10 &	Pcity &	人员今日途经城市 &	Char &	30 & 非空 \\
	11  &	Ptemp &	配送人员体温 &	Float &	4 &	非空 \\
	12 &	Pupdate &	健康信息更新时间 &	Date &	3 &	非空 \\
	13 &	Ono &	商品订单编号 &	Integer &	4 &	非空,商品订单信 \\ &&&&& 息表主码\\
	14 &	Otime &	商品订单时间 &	Date &	3 &	非空 \\
	15 &	Iname 	& 货品名称 &	Char &	30 &	非空 \\
	16 &	Ino &	货品编号 &	Integer &	4 &	非空 \\
	17 &	Inum &	订单货品数额 &	Integer &	4 &	非空 \\
	18 &	Tname &	物流公司名称 &	Char &	30 	& 非空 \\
	19 &	Tno &	物流公司编号 &	 Integer &	4 &	非空 \\
	20 & Tscore & 物流公司评分 & Float &	4 &	非空 \\
	21 & Tdate & 物流公司入驻平台 & Date & 3 & 非空 \\ && 时间 &&& \\
	22 &	Ovalue 	& 商品订单金额 	& Float &	4 &	非空 \\
	23 & Dno &	物流单号 &	Integer &	4 &	非空,物流订单信 \\ &&&&& 息表主码 \\
	24 &	Dtrans &	物流运送方式 &	Char &	30 &	非空 \\
	25 &	Dloc &	物流所在位置 &	Integer &	4 &	非空 \\
	26 &	Dupdate &	物流更新时间 &	Date &	3 &	非空 \\
	27 &	Dstate &	物流风险信息 &	Tinyint &	1 &	非空,0异常; \\ &&&&& 1正常\\
	28 & Dsetime & 发货时间 & Date & 3 & 非空 \\
	29 & Dretime & 收货时间 & Date & 3 & 非空 \\
	30 &	Cname &	城市名称 &	Char &	30 	& 非空 \\
	31 &	Cno &	城市编号 &	Integer &	4 &	非空 \\
	32 &	Cstate &	城市疫情状况 &	Tinyint &	1 &	 非空,0、1、2分\\ &&&&& 别代表低中高风险 \\
	33 &	Cloc &	城市坐标 &	Float &	4 &	非空 \\
	34 &	Cupdate &	城市疫情更新时间 &	Date &	3 &	非空 \\
	35 &	Dvalue 	& 物流订单金额 	& Float &	4 &	非空 \\
	36 &	Zpeople & 配送人员风险评分 & Integer & 4 & 非空 \\
	37 &	Zloc & 配送路线风险评分 & Integer & 4 & 非空 \\
	38 &	Dpno & 订单分配编号 & Integer & 4 & 非空 \\
	39 &	Otype & 商品订单类型 & Char &	30 	& 非空 \\
	40 &	Onum & 商品订单数目 & Integer &	4 &	非空 \\
\end{supertabular}
\end{center}

\subsubsection{数据结构}

注:\underline{下划线}表示主码,\textit{斜体}表示外码。
\vspace{0.2cm}
\tablehead{
	\toprule
	\textbf{编号} & \textbf{数据结构} & \textbf{含义说明} & \multicolumn{3}{l}{\textbf{组成}} \\
	\midrule}
\tabletail{\bottomrule}

\begin{center}
\begin{supertabular}{llllll}
1 &	Buyer User &	买方用户信息表 & Uname &	\uline{Uno} &	Uaddress  \\ & Information &&	Uphone && \\
2 &	Seller User &	卖方用户信息表 & Sname &	\uline{Sno} &	Saddress \\ & Information &&&& \\
3 &	Order  & 商品订单信息表 &	\uline{Ono} &	Otime &	Ovalue  \\
& Information &&	\textit{Uno} &	\textit{Sno}  & Otype \\ &&& Onum & &\\
4 &	Delivery &	物流订单信息表 &	\uline{Dno} &	Dvalue &	Dtrans  \\ & Information && Tno &	\textit{Sno} & Dsetime \\ &&& Dretime & \textit{Ono} & \\
5 &	Staff Health &	员工健康信息表 &	\uline{Pno} &	Pcity & Ptemp \\ 
& Information && \uline{Pupdate} && \\
6 &	Pandemic &	疫情情况信息表 &	Cname &	\uline{Cno} &	Cstate \\	& Information &&  Cloc & \uline{Cupdate}	& \\
7 &Staff  &	物流人员信息表 &	Pname &	\uline{Pno} &	Tname \\ & Information	& & Tno & Tscore & Tdate \\
8 &	Geographic  &	物流位置信息表 &	\uline{Dno} &	Dloc &	\uline{Dupdate} \\& Information&&&& \\
9 & Summary & 物流订单总表 & \uline{Ono} & \textit{Dno} & Zpeople \\ & Information && Zloc && \\
10 & Delivery & 物流订单分配表 & \uline{Dpno} & \textit{Dno} & \textit{Pno}\\ & Distribution &&&& \\
\end{supertabular}
\end{center}

\subsubsection{数据流}

\tablehead{
	\toprule
	\textbf{编号} & \textbf{数据流名} & \textbf{来源} & \textbf{去向} &  \textbf{组成} \\
	\midrule}
\tabletail{\bottomrule}

\begin{center}
\begin{supertabular}{lllll}
	1 & 买方用户信息 &	电商数据同步 &	数据统计 &	买方姓名,联系方式地址等 \\&&& 与可视化 & \\
	2 &	卖方用户信息 & 电商数据同步 &	数据统计 &	卖方姓名,联系方式地址等\\&&& 与可视化 & \\
	3 &	商品订单信息 & 交易 &	提交物流 &	商品订单号,商品名称、 \\ &&&订单& 数量、金额等 \\
	4 &	疫情情况信息 &	官方数据爬取	& 识别物流 &	全国各地区疫情风险情况 \\ &&& 风险 & (每日更新)\\
	5 &	员工健康信息 &	自测健康状况 &	安排配送、 &	员工体温等健康信息 \\ &&& 识别物流 &(每日更新)\\ &&& 风险 &\\
	6 &	物流人员信息 &	公司数据同步 &	安排配送 &	员工编号、姓名等信息 \\
	7 &	物流订单信息 &	提交物流订单	& 安排配送 & 	物流订单号,物流运送方\\ &&&& 式,运费等信息 \\
	8 &	物流位置信息 & 安排配送 &	识别物流 &	物流当前所在位置及所经 \\ &&& 风险 & 过的路径信息 \\
	9 & 物流订单总表 & 识别物流风险 & 数据统计 & 物流疫情相关信息汇总 \\ &&  & 与可视化 & \\
\end{supertabular}
\end{center}

\subsubsection{数据储存}

\tablehead{
	\toprule
	\textbf{编号} & \textbf{数据存储名} & \textbf{输入} & \textbf{输出} &  \textbf{说明} \\
	\midrule}
\tabletail{\bottomrule}

\begin{center}
\begin{supertabular}{lllll}
	1 &	买方用户信息表 &	买家用户信息 	& 无 &	存储所有买家用户信息 \\
	2 &	卖方用户信息表 &	卖家用户信息 &	无 &	存储所有卖家用户信息 \\
	3 &	商品订单信息表 &	商品订单信 &	订单信息 &	存储所有商品订单相关 \\ && 息,价格 && 信息 \\ 
	4 &	物流订单信息表 &	物流信息价 &	物流信息 &	存储所有物流订单的相 \\ && 格,运输方式 && 关信息 \\
	5 &	员工健康信息表 &	配送人员信 &	健康状况 &	存储配送人员每日更新 \\ && 息,健康状况 && 的健康状况信息 \\
	6 &	疫情情况信息表 &	每日国内疫 &	疫情情况 &	存储国内各地区每日疫 \\ && 情情况 & 可视化 & 情情况更新的信息 \\
	7 &	物流人员信息表 &	员工名单 &	无 &	存储物流公司配送人员 \\ &&&& 名单 \\
	8 & 物流位置信息表 &	定时更新物 &	物流路线 &	存储物流位置实时信息 \\ && 流位置坐标 & 可视化 & \\
	9 & 物流订单总表 & 物流单号,& 数据统计 & 存储物流与订单关系\\ && 订单编号 & 与可视化 & \\
\end{supertabular}
\end{center}

\subsubsection{处理过程}

\tablehead{
	\toprule
	\textbf{编号} & \textbf{处理过程名} & \textbf{输入数据流} & \textbf{输出数据流 } &  \textbf{加工逻辑} \\
	\midrule}
\tabletail{\bottomrule}

\begin{center}
\begin{supertabular}{lllll}
	1 &	买方上传信息 &	买方用户信息	& 买方信息查询 &	由买方授权,从电商 \\ &&&& 平台接口进行导入 \\
	2 &	卖方上传信息 &	卖方用户信息 &	卖方信息查询 & 由卖方授权,从电商 \\ &&&& 平台接口进行导入 \\
	3 &	电商平台上传 &	商品订单信息 &	订单查询 &	由电商平台上传订单 \\ & 信息 &&& 信息与买卖关系 \\
	4 & 卖方提交订单 & 物流订单信息  &	物流查询  &	由卖方提交物流信息 \\
	5 &	物流公司上传 &	物流人员信息 &	物流查询、物 &	由物流公司进行统计 \\ & 信息 & & 流人员查询 & 上传 \\
	6 &	配送人员上传 &	每日健康状况 &	配送人员健康 &	由配送人员每日上传 \\ & 信息 & 信息 & 信息 & 当日健康状况信息 \\
	7 &	配送人员健康 &	申请查询配送 &	配送人员健康 &	根据配送人员每日健 \\ & 状况查询 & 人员健康信息 & 信息 & 康状况的统计数据,\\ &&&& 进行查询,检查配送 \\ &&&& 人员健康自检是否满 \\ &&&& 足要求 \\
	8 &	疫情统计 &	爬取城市地区 &	疫情统计可视 &	爬取官方发布的疫情 \\ && 疫情分布情况 & 化结果 & 分布情况,分析地理 \\ && 及坐标 && 位置坐标标注 \\
	9 &	路线查询 &	查询订单坐标 &	坐标路径可视 &	根据物流坐标的信 \\ && 信息 & 化结果 & 息,生成物流路径供 \\ &&&& 买方查看配送进度 \\
	10 &	物流疫情情况 &	路径可视化结 & 疫情-物流图 &	根据疫情与物流坐 \\ & 查询 & 果、疫情统计 && 标,合成疫情-物流 \\ && 可视化结果 && 可视化的图像,并对 \\ &&&& 路径疫情情况进行 \\ &&&& 评估 \\
\end{supertabular}
\end{center}

\subsection{安全性和完整性要求}

\subsubsection{安全性要求}

平台系统的使用者主要分为物流公司、电商平台、卖家、买家和游客。其中,物流公司、电商平台、卖家和买家在使用平台服务前,需要经过注册成为系统用户,在数据库内储存其设置的用户信息,并通过系统核实认证身份,保证身份不被盗用,以确保用户信息的真实性和数据安全性。未经系统核实认证身份的游客只行使浏览权限,且只能够浏览平台提供服务介绍信息和中高风险地区总地图等相关信息。游客无需向系统输入会保留在数据库中的内容,也无法查询超出其游客权限的内容。

由于系统涉及个人隐私数据(如买家卖家基本信息等)和商业敏感性数据(如订单信息,商家成本收益等),为了保护各方隐私安全,系统对于不同的认证用户身份设置了不同的访问权限。对于物流公司身份用户,只能够查看由其提供的物流人员健康信息、包裹配送信息进行数据统计与可视化之后的报告,并对相应信息进行查询;电商平台身份用户只能够查看由其提供订单信息进行数据处理与可视化之后的报告,并进行相应查询;买家用户和卖家用户的个人隐私数据只有本人有权限查看,且只能查询和本人相关的订单及产品包裹配送信息;游客没有任何查询个人隐私数据和商业敏感数据的权限。对于不涉及系统核心业务的数据(如商家成本收益),数据提供者有权选择不提供此类信息。对于任何分析报告中统计性数据,将对涉及订单、用户信息进行匿名化处理,绝不公开个体数据。

\subsubsection{完整性要求}

\noindent \textbf{(1) 实体完整性}
\par
\vspace{0.1cm}
\noindent 作为主码的属性不能取空值且取值唯一,详见数据字典数据项部分的限制条件。

\noindent \textbf{(2) 参照完整性}
\par
\vspace{0.1cm}
\noindent 作为外码的属性或者为空值或者为被参照关系中某个元组的主码值,详见数据字典数据项部分的限制条件。

\noindent \textbf{(3) 用户定义的完整性}
\par
\vspace{0.1cm}
\noindent 根据系统的具体功能,某些属性不能取空值或者对属性的取值范围有限制,详见数据字典数据项部分的限制条件。

\subsection{需求分析过程中的经验教训总结}

随着疫情逐渐常态化,人们对于实时监控周边疫情信息的需求越发迫切。于是,对于作为最容易成为密切接触者,又是人们生活中最常见的物流配送人员的健康状况和包裹近期途径地风险的监控,便成为防控疫情的重中之重。然而,据我们问卷调查显示,在大部分买家看来,现在大多数主流电商平台和物流公司在对物流配送人员和包裹的相关疫情信息监控上所做出的努力还远远不够。信息滞后和防控力度不足亟需社会推出一个新的“疫情+物流”信息整合平台。于是我们组尝试在这方面做出努力,搭建一个能够实现疫情及物流订单数据储存、整合并动态可视化呈现的平台,以同时满足物流公司、电商平台、买家和卖家对于疫情下物流配送及包裹信息的监控需求。由于我们的问卷调查覆盖人群及调研时间有限,问卷所得的结果与实际情况可能存在偏差,这对我们的需求分析可能会造成一定的影响,若时间及条件允许,我们会尝试开展更深入的调查以获得更准确的结论。

\section{概念结构设计}

\subsection{局部分 E-R 图 }

\subsubsection{提交物流订单}

\begin{center}
	\includegraphics[height = 17cm]{e-r-part-1.png}
\end{center}

\subsubsection{安排配送}

\begin{center}
	\includegraphics[height = 15.5cm]{e-r-part-2.png}
\end{center}

\subsubsection{识别物流风险}

\begin{center}
	\includegraphics[height = 17cm]{e-r-part-3.png}
\end{center}

\subsection{全局 E-R 图 }

\subsubsection{不包含属性的 E-R 图}

\begin{center}
	\includegraphics[height = 15.8cm]{e-r-total.png}
\end{center}

\subsubsection{每个实体和联系的属性描述}

\noindent \textbf{(1) 卖方}

\begin{center}
	\includegraphics[height = 3cm]{e-r-e1.png}
\end{center}

\noindent \textbf{(2) 物流公司}

\begin{center}
	\includegraphics[height = 3.5cm]{e-r-e2.png}
\end{center}

\noindent \textbf{(3) 物流订单}

\begin{center}
	\includegraphics[height = 9.5cm]{e-r-e3.png}
\end{center}

\noindent \textbf{(4) 物流人员}

\begin{center}
	\includegraphics[height = 2.5cm]{e-r-e4.png}
\end{center}


\noindent \textbf{(5) 物流位置}

\begin{center}
	\includegraphics[height = 2.5cm]{e-r-e5.png}
\end{center}

\newpage
\noindent \textbf{(6) 疫情情况}

\begin{center}
	\includegraphics[height = 7cm]{e-r-e6.png}
\end{center}

\noindent \textbf{(7) 员工健康信息}

\begin{center}
	\includegraphics[height = 7cm]{e-r-e7.png}
\end{center}

\noindent \textbf{(8) 联系:分配}

\begin{center}
	\includegraphics[height = 1.4cm]{e-r-r3.png}
\end{center}

\noindent \textbf{(9) 联系:匹配1}

\begin{center}
	\includegraphics[height = 1.4cm]{e-r-r1.png}
\end{center}

\noindent \textbf{(10) 联系:匹配2}

\begin{center}
	\includegraphics[height = 1.4cm]{e-r-r2.png}
\end{center}

\section{逻辑结构设计}

\subsection{E-R 图向关系模型的转换过程}

\subsubsection{提交物流订单}

\noindent \textbf{(1) 卖方} \par 
\noindent Sno $\rightarrow$ Sname, Sno $\rightarrow$ Saddress \par 
\noindent 候选码:Seller (\uline{Sno}, Sname, Saddress) 

\begin{center}
	\includegraphics[height = 3cm]{relation-model-1.png}
\end{center}

\noindent \textbf{(2) 物流订单} \par 
\noindent Dno $\rightarrow$ Dvalue, Dno $\rightarrow$ Dtrans, Dno $\rightarrow$ Tno, Dno $\rightarrow$ Sno \\
\noindent Dno $\rightarrow$ Dsetime, Dno $\rightarrow$ Dretime \par 
\noindent 候选码:Delivery (\uline{Dno}, Dvalue, Dtrans, Tno, Sno, Dsetime, Dretime)

\begin{center}
	\includegraphics[height = 8cm]{relation-model-2.jpg}
\end{center}

\noindent \textbf{(3) 物流公司} \par 
\noindent Tno $\rightarrow$ Tname, Tno $\rightarrow$ Tscore,Tno $\rightarrow$ Tdate \par 
\noindent 候选码:Transportation Company (\uline{Tno}, Tname, Tscore, Tdate)

\begin{center}
	\includegraphics[height = 4.5cm]{relation-model-3.png}
\end{center}

\subsubsection{安排配送}

\noindent \textbf{(1) 物流订单 (略)} \par 

\noindent \textbf{(2) 分配} \par 
\noindent (Tno, Pno) $\rightarrow$ Dpno \par
\noindent 候选码:Acceptance (\uline{Tno}, \uline{Pno}, Dpno)

\begin{center}
	\includegraphics[height = 3cm]{relation-model-4.png}
\end{center}

\noindent \textbf{(3) 物流人员} \par 
\noindent Pno $\rightarrow$ Pname, Pno $\rightarrow$ Tno \par
\noindent 候选码:Staff (\uline{Pno}, Pname, Tno)

\begin{center}
	\includegraphics[height = 3cm]{relation-model-5.png}
\end{center}

\noindent \textbf{(4) 物流位置} \par 
\noindent (Dno, Dupdate) $\rightarrow$ Dloc, (Dno, Dupdate) $\rightarrow$ Pno \par
\noindent 候选码:Geography (\uline{Dno}, \uline{Dupdate}, Dloc, Pno)

\begin{center}
	\includegraphics[height = 3cm]{relation-model-6.png}
\end{center}

\subsubsection{识别物流风险}

\noindent \textbf{(1) 物流位置} \par 
\noindent (Dno, Dupdate) $\rightarrow$ Dloc, (Dno, Dupdate) $\rightarrow$ Zloc,\\ 
\noindent (Dno, Dupdate) $\rightarrow$ Cno, (Dno, Dupdate) $\rightarrow$ Cupdate \par
\noindent 候选码:Geography (\uline{Dno}, \uline{Dupdate}, Dloc, Zloc, Cno, Cupdate)

\begin{center}
	\includegraphics[height = 5.5cm]{relation-model-7.png}
\end{center}

\noindent \textbf{(2) 疫情情况} \par 
\noindent (Cno, Cupdate) $\rightarrow$ Cname, (Cno, Cupdate) $\rightarrow$ Cstate,\\ \noindent (Cno, Cupdate) $\rightarrow$ Cloc \par
\noindent 候选码:Pandemic (\uline{Cno}, \uline{Cupdate}, Cname, Cstate, Cloc)

\begin{center}
	\includegraphics[height = 5.5cm]{relation-model-8.png}
\end{center}

\noindent \textbf{(3) 匹配2} \par 
\noindent (Cno, Cupdate, Pno, Pupdate) $\rightarrow$ Zpeople \par
\noindent 候选码:Match2 (\uline{Cno}, \uline{Cupdate}, \uline{Pno}, \uline{Pupdate}, Zpeople)

\begin{center}
	\includegraphics[height = 6cm]{relation-model-9.png}
\end{center}

\noindent \textbf{(4) 员工健康信息} \par 
\noindent (Pno, Pupdate) $\rightarrow$ Pcity, (Pno, Pupdate) $\rightarrow$ Ptemp \par
\noindent 候选码:Staff Health (\uline{Pno}, \uline{Pupdate}, Pcity, Ptemp)

\begin{center}
	\includegraphics[height = 3cm]{relation-model-10.png}
\end{center}


\subsection{关系模型的优化过程}

\subsubsection{提交物流订单}

\noindent \textbf{(1) 卖方} \par 
\noindent \textbf{关系模式:} Seller(\uline{Sno}, Sname, Saddress) \par 
\noindent \textbf{范式:} Seller $\in$ BCNF \par 
\begin{itemize}
	\item 关系模式Seller(\uline{Sno}, Sname, Saddress)只有一个码Sno,没有任何属性对Sno部分依赖或传递依赖,所以Seller $\in$ 3NF
	\item 同时Seller中Sno是唯一决定因素,所以Seller $\in$ BCNF
\end{itemize}
\textbf{函数依赖:} Sno $\rightarrow$ Sname, Sno $\rightarrow$ Saddress

\vspace{0.3cm}
\noindent \textbf{(2) 物流订单} \par 
\noindent \textbf{关系模式:} Delivery(\uline{Dno}, Dvalue, Dtrans, Tno, Sno, Dsetime, Dretime) \par 
\noindent \textbf{范式:} Delivery $\in$ BCNF \par 
\begin{itemize}
	\item 关系模式Delivery(\uline{Dno}, Dvalue, Dtrans, Tno, Sno, Dsetime, Dretime)只有一个码Dno,没有任何属性对Dno部分依赖或传递依赖,所以Delivery $\in$ 3NF
	\item 同时Delivery中Dno是唯一决定因素,所以Delivery $\in$ BCNF
\end{itemize}
\textbf{函数依赖:} Dno $\rightarrow$ Dvalue, Dno $\rightarrow$ Dtrans, Dno $\rightarrow$ Tno, Dno $\rightarrow$ Sno, Dno $\rightarrow$ Dsetime, Dno $\rightarrow$ Dretime

\vspace{0.3cm}
\noindent \textbf{(3) 物流公司} \par 
\noindent \textbf{关系模式:} Transportation Company(\uline{Tno}, Tname,  Tscore, Tdate) \par 
\noindent \textbf{范式:} Transportation Company $\in$ BCNF \par 
\begin{itemize}
	\item 关系模式Transportation Company(\uline{Tno}, Tname, Tscore, Tdate)只有一个码Tno,没有任何属性对Tno部分依赖或传递依赖,所以Transportation Company $\in$ 3NF
	\item 同时Transportation Company中Tno是唯一决定因素,所以Transportation Company $\in$ BCNF
\end{itemize}
\textbf{函数依赖:} Tno $\rightarrow$ Tname, Tno $\rightarrow$ Tscore, Tno $\rightarrow$ Tdate

\subsubsection{安排配送}
\noindent \textbf{(1) 物流订单 (略)} 

\vspace{0.3cm}
\noindent \textbf{(2) 分配} \par 
\noindent \textbf{关系模式:} Acceptance(\uline{Tno}, \uline{Pno}, Dpno) \par 
\noindent \textbf{范式:} Acceptance $\in$ BCNF \par 
\begin{itemize}
	\item 关系模式Acceptance(\uline{Tno}, \uline{Pno}, Dpno)有Tno和Pno两个码,这两个码都由单个属性组成,彼此不相交
	\item 其他属性不存在对码的部分依赖或传递依赖,所以Acceptance $\in$ 3NF
	\item 同时Acceptance中除Tno, Pno外没有其他决定因素,所以Acceptance $\in$ BCNF
\end{itemize}
\textbf{函数依赖:} (Tno, Pno) $\rightarrow$ Dpno

\vspace{0.3cm}
\noindent \textbf{(3) 物流人员} \par 
\noindent \textbf{关系模式:} Staff(\uline{Pno}, Pname, Tno) \par 
\noindent \textbf{范式:} Staff $\in$ BCNF \par 
\begin{itemize}
	\item 关系模式Staff(\uline{Pno}, Pname, Tno)只有一个码Pno,没有任何属性对Pno部分依赖或传递依赖,所以Staff $\in$ 3NF
	\item 同时Staff中Pno是唯一决定因素,所以Staff $\in$ BCNF
\end{itemize}
\textbf{函数依赖:} Pno $\rightarrow$ Pname, Pno $\rightarrow$ Tno

\vspace{0.3cm}
\noindent \textbf{(4) 物流位置} \par 
\noindent \textbf{关系模式:} Geography(\uline{Dno}, \uline{Dupdate}, Dloc, Pno) \par 
\noindent \textbf{范式:} Geography $\in$ BCNF \par 
\begin{itemize}
	\item 关系模式Geography(\uline{Dno}, \uline{Dupdate}, Dloc, Pno)有Dno和Dupdate两个码,这两个码都由单个属性组成,彼此不相交
	\item 其他属性不存在对码的部分依赖或传递依赖,所以Geography $\in$ 3NF
	\item 同时Geography中除Dno, Dupdate外没有其他决定因素,所以Geography $\in$ BCNF
\end{itemize}
\textbf{函数依赖:} (Dno, Dupdate) $\rightarrow$ Dloc, (Dno, Dupdate) $\rightarrow$ Pno

\subsubsection{识别物流风险}

\noindent \textbf{(1) 物流位置} \par 
\noindent \textbf{关系模式:} Geography(\uline{Dno}, \uline{Dupdate}, Dloc, Zloc, Cno, Cupdate) \par 
\noindent \textbf{范式:} Geography $\in$ BCNF \par
\begin{itemize}
	\item 关系模式Geography(\uline{Dno}, \uline{Dupdate}, Dloc, Zloc, Cno, Cupdate)有Dno和Dupdate两个码,这两个码都由单个属性组成,彼此不相交
	\item 其他属性不存在对码的部分依赖或传递依赖,所以Geography $\in$ 3NF
	\item 同时Geography中除Dno, Dupdate外没有其他决定因素,所以Geography $\in$ BCNF
\end{itemize} 
\textbf{函数依赖:} (Dno, Dupdate) $\rightarrow$ Dloc, (Dno, Dupdate) $\rightarrow$ Zloc, (Dno, Dupdate) $\rightarrow$ Cno, (Dno, Dupdate) $\rightarrow$ Cupdate

\vspace{0.3cm}
\noindent \textbf{(2) 疫情情况} \par 
\noindent \textbf{关系模式:} Pandemic(\uline{Cno}, \uline{Cupdate}, Cname, Cstate, Cloc) \par 
\noindent \textbf{范式:} Pandemic $\in$ BCNF \par 
\begin{itemize}
	\item 关系模式Pandemic(\uline{Cno}, \uline{Cupdate}, Cname, Cstate, Cloc)有Cno和Cupdate两个码,这两个码都由单个属性组成,彼此不相交
	\item 其他属性不存在对码的部分依赖或传递依赖,所以Pandemic $\in$ 3NF
	\item 同时Pandemic中除Cno, Cupdate外没有其他决定因素,所以Pandemic $\in$ BCNF
\end{itemize}
\textbf{函数依赖:} (Cno, Cupdate) $\rightarrow$ Cname, (Cno, Cupdate) $\rightarrow$ Cstate, (Cno, Cupdate) $\rightarrow$ Cloc

\vspace{0.3cm}
\noindent \textbf{(3) 匹配2} \par 
\noindent \textbf{关系模式:} Match2(\uline{Cno}, \uline{Cupdate}, \uline{Pno}, \uline{Pupdate}, Zpeople) \par 
\noindent \textbf{范式:} Match2 $\in$ BCNF \par 
\begin{itemize}
	\item 关系模式Match2(\uline{Cno}, \uline{Cupdate}, \uline{Pno}, \uline{Pupdate}, Zpeople)有Cno, Cupdate, Pno, Pupdate四个码,这四个码都由单个属性组成,彼此不相交
	\item 其他属性不存在对码的部分依赖或传递依赖,所以Match2 $\in$ 3NF
	\item 同时Match2中除Cno, Cupdate, Pno, Pupdate外没有其他决定因素,所以Match2 $\in$ BCNF
\end{itemize}
\textbf{函数依赖:} (Cno, Cupdate, Pno, Pupdate) $\rightarrow$ Zpeople

\vspace{0.3cm}
\noindent \textbf{(4) 员工健康信息} \par 
\noindent \textbf{关系模式:} Staff Health(\uline{Pno}, \uline{Pupdate}, Pcity, Ptemp) \par 
\noindent \textbf{范式:} Staff Health $\in$ BCNF \par
\begin{itemize}
	\item 关系模式Staff Health(\uline{Pno}, \uline{Pupdate}, Pcity, Ptemp)有Pno和Pupdate两个码,这两个码都由单个属性组成,彼此不相交
	\item 其他属性不存在对码的部分依赖或传递依赖,所以Staff Health $\in$ 3NF
	\item 同时Staff Health中除Pno, Pupdate外没有其他决定因素,所以Staff Health $\in$ BCNF
\end{itemize} 
\textbf{函数依赖:} (Pno, Pupdate) $\rightarrow$ Pcity, (Pno, Pupdate) $\rightarrow$ Ptemp

\subsection{最终的关系模式}

\noindent \textbf{(1)} Seller(\uline{Sno}, Sname, Saddress)

\begin{center}
	\begin{tabular}{lllll}
		\toprule
		Sno &	卖方编号 &	Integer &	4 &	非空,卖方信息表主码 \\
		Sname &	卖方名称 &	Char &	30 &	非空 \\
		Saddress &	卖方仓储位置 &	Char &	30 &	非空 \\
		\bottomrule
	\end{tabular}
\end{center}

\vspace{0.5cm}
\noindent \textbf{(2)} Delivery(\uline{Dno}, Dvalue, Dtrans, Tno, Sno, Dsetime, Dretime)

\begin{center}
	\begin{tabular}{lllll}
		\toprule
		Dno &	物流单号 &	Integer &	4 &	非空 \\
		Dvalue &	物流订单金额 &	Float &	4 &	非空 \\
		Dtrans &	物流运送方式 &	Char &	30 &	非空 \\
		Tno &	货运公司编号 &	Integer &	4 &	非空 \\
		Sno &	卖方编号 &	Integer &	4 &	非空 \\
		Dsetime & 发货时间 & Date & 3  & 非空 \\
		Dretime & 收货时间 & Date & 3  & 非空 \\
		\bottomrule
	\end{tabular}
\end{center}

\vspace{0.5cm}
\noindent \textbf{(3)} Transportation Company(\uline{Tno}, Tname, Tscore, Tdate)

\begin{center}
	\begin{tabular}{lllll}
		\toprule
		Tno &	货运公司编号 &	Integer &	4 &	非空 \\
		Tname &	货运公司名称 &	Char &	30 &	非空 \\
		Tscore & 货运公司评分 & Integer & 4 & 非空 \\
		Tdate & 公司入驻日期 & Date  & 3 & 非空 \\
		\bottomrule
	\end{tabular}
\end{center}

\vspace{0.5cm}
\noindent \textbf{(4)} Acceptance(\uline{Tno}, \uline{Pno}, Dpno)

\begin{center}
	\begin{tabular}{lllll}
		\toprule
		Tno &	货运公司编号 &	Integer &	4 &	非空 \\
		Pno &	配送人员编号 &	Integer &	4 &	非空 \\
		Dpno &	配送订单号 &	Integer &	4 &	非空 \\
		\bottomrule
	\end{tabular}
\end{center}

\vspace{0.5cm}
\noindent \textbf{(5)} Staff(\uline{Pno}, Pname, Tno)

\begin{center}
	\begin{tabular}{lllll}
		\toprule
		Pno &	配送人员编号 &	Integer &	4 &	非空 \\
		Pname &	配送人员姓名 &	Char &	30 &	非空 \\
		Tno &	货运公司编号 &	Integer &	4 &	非空 \\
		\bottomrule
	\end{tabular}
\end{center}

\vspace{0.5cm}
\noindent \textbf{(6)} Geography(\uline{Dno}, \uline{Dupdate}, Dloc, Pno)

\begin{center}
	\begin{tabular}{lllll}
		\toprule
		Dno &	物流单号 &	Integer &	4 &	非空 \\
		Dupdate &	物流更新时间 &	Date &	3 &	非空 \\
		Dloc &	物流所在位置 &	Integer &	4 &	非空 \\
		Pno &	配送人员编号 &	Integer &	4 &	非空 \\
		\bottomrule
	\end{tabular}
\end{center}

\vspace{0.5cm}
\noindent \textbf{(7)} Geography(\uline{Dno}, \uline{Dupdate}, Dloc, Zloc, Cno, Cupdate)

\begin{center}
	\begin{tabular}{lllll}
		\toprule
		Dno &	物流单号 &	Integer &	4 &	非空 \\
		Dupdate &	物流更新时间 &	Date &	3 &	非空 \\
		Dloc &	物流所在位置 &	Integer &	4 &	非空 \\
		Zloc &	配送路线风险评分 &	Integer &	4 &	非空 \\
		Cno &	城市编号 &	Integer &	4 &	非空 \\
		Cupdate &	城市疫情更新时间 &	Date &	3 &	非空 \\
		\bottomrule
	\end{tabular}
\end{center}

\vspace{0.5cm}
\noindent \textbf{(8)} Pandemic(\uline{Cno}, \uline{Cupdate}, Cname, Cstate, Cloc)

\begin{center}
	\begin{tabular}{lllll}
		\toprule
		Cno &	城市编号 &	Integer &	4 &	非空 \\
		Cupdate &	城市疫情更新时间 &	Date &	3 &	非空 \\
		Cname &	城市名称 &	Char &	30 &	非空 \\
		Cstate &	城市疫情状况 &	Tinyint &	1 &	非空,0、1、2分别代表 \\ &&&& 低中高风险 \\
		Cloc &	城市坐标 &	Float &	4 &	非空 \\
		\bottomrule
	\end{tabular}
\end{center}

\vspace{0.5cm}
\noindent \textbf{(9)} Match2(\uline{Cno}, \uline{Cupdate}, \uline{Pno}, \uline{Pupdate}, Zpeople)

\begin{center}
	\begin{tabular}{lllll}
		\toprule
		Cno &	城市编号 &	Integer &	4 &	非空 \\
		Cupdate &	城市疫情更新时间 &	Date &	3 &	非空 \\
		Pno &	配送人员编号 &	Integer &	4 &	非空 \\
		Pupdate &	人员体温更新时间 &	Date &	3 &	非空 \\
		Zpeople &	配送人员风险评分 &	Integer &	4 &	非空 \\
		\bottomrule
	\end{tabular}
\end{center}

\vspace{0.5cm}
\noindent \textbf{(10)} Staff Health(\uline{Pno}, \uline{Pupdate}, Pcity, Ptemp)

\begin{center}
	\begin{tabular}{lllll}
		\toprule
		Pno &	配送人员编号 &	Integer &	4 &	非空 \\
		Pupdate &	人员体温更新时间 &	Date &	3 &	非空 \\
		Pcity &	配送人员今日途经城市 &	Char &	30 &	非空 \\
		Ptemp &	配送人员体温 &	Float &	4 &	非空 \\
		\bottomrule
	\end{tabular}
\end{center}

\subsection{设计用户的子模式}

\subsubsection{电商平台的查询视图:商品订单-疫情相关信息报表}

\noindent \textbf{(1)商品订单的基本信息} \par 
\noindent Application-Ecommerce1(卖方编号,卖方名称,卖方仓储位置,物流单号,物流订单金额,物流运送方式,货运公司名称,发货时间,收货时间,物流更新时间,物流所在位置,配送人员姓名)

\subsubsection{物流公司的查询视图:物流运输-疫情相关信息报表}

\noindent \textbf{(1)物流订单的基本信息} \par 
\noindent Application-Delivery1(卖方编号,卖方名称,卖方仓储位置,物流单号,物流订单金额,物流运送方式,货运公司编号,货运公司名称,配送订单号,配送人员编号,配送人员姓名,发货时间,收货时间,物流更新时间,物流所在位置,配送路线风险评分)

\vspace{0.3cm}
\noindent \textbf{(2)城市的疫情相关信息} \par 
\noindent Application-Delivery2(城市编号,城市名称,城市疫情状况,城市疫情更新时间,经过该城市的配送人员编号,配送人员经过该城市的时间,经过该城市的物流单号,物流订单经过该城市的时间)

\vspace{0.3cm}
\noindent \textbf{(3)配送人员的基本信息} \par 
\noindent Application-Delivery3(配送人员编号,配送人员姓名,配送人员体温,人员体温更新时间,配送人员途经城市,配送人员途经城市疫情情况,途径城市疫情更新时间,配送人员配送过/正在配送的物流订单号)

\subsubsection{卖家的查询视图:卖家订单-物流相关信息报表}

\noindent \textbf{(1)卖家订单的基本信息} \par 
\noindent Application-Seller1(卖方编号,卖方名称,卖方仓储位置,物流单号,物流订单金额,物流运送方式,货运公司名称,发货时间,收货时间,物流更新时间,物流所在位置)

\vspace{0.3cm}
\noindent \textbf{(2)卖家订单的疫情相关信息} \par 
\noindent Application-Seller2(物流单号,物流运送方式,货运公司名称,配送人员名称,配送人员体温,人员体温更新时间,配送人员途经城市,途径城市疫情更新时间)

\subsubsection{买家的查询视图:买家订单-物流相关疫情信息报表}

\noindent \textbf{(1)买家订单的基本信息} \par 
\noindent Application-Buyer1(卖方名称,卖方仓储位置,物流单号,物流订单金额,物流运送方式,货运公司名称,发货时间,收货时间,物流更新时间,物流所在位置,配送人员姓名,配送人员体温)

\vspace{0.3cm}
\noindent \textbf{(2)买家订单的疫情相关信息} \par 
\noindent Application-Buyer2(卖方名称,卖方仓储位置,物流单号,物流订单金额,物流运送方式,货运公司名称,发货时间,收货时间,物流更新时间,物流所在位置,配送人员姓名,配送人员体温,配送人员经过的城市,配送人员经过该城市的时间,配送人员经过城市的疫情状况)

\newpage
\section{数据库实施及程序演示}
\subsection{环境和配置}

我们小组基于Python中的Django框架实现数据库管理系统。Django的backend采用MySQL,网页设计采用html + CSS + JavaScript。此外,我们小组通过Python中的Scrapy框架实现了能够采集疫情数据的网络爬虫。主要的环境配置总结如下。

\begin{center}
	\begin{tabular}{lllll}
	\toprule
	Item & Python & MySQL & Django & Scrapy \\
	\midrule
	Version & 3.8 & 8.0 & 3.2 & 2.4\\
	\bottomrule
\end{tabular}
\end{center}

为了提高效率,我们小组通过Git + GitHub合作编写程序。项目源代码可直接在GitHub网站上查看:\url{https://github.com/ShupeiLi/database-project} 。

\subsubsection{Python第三方库说明}
为实现特定功能,我们小组使用了一些Python第三方库,详细说明汇总如下。

\begin{center}
\begin{tabular}{lll}
\toprule
\textbf{第三方库}  & \textbf{经测试版本} & \textbf{主要用途} \\
\midrule
Django & 3.2.9 & 网站框架搭建 \\
django-bootstrap-v5 & 1.0.7 & 网页UI美化 \\
django-filter & 21.1 & 实现各类筛选器 \\
Faker & 10.0.0 & 生成随机姓名、地址等 \\
mysql-connector-python & 8.0.27 & MySQL的Python接口 \\
openpyxl & 3.0.9 & 读取 .xlsx文件 \\
pandas & 1.3.3 & 数据框处理 \\
phone-gen & 2.1.7 & 生成随机电话号码 \\
Scrapy & 2.5.1 & 网络爬虫框架 \\
selenium & 4.1.0 & 运行爬虫时模拟浏览器 \\ 
SQLAlchemy & 1.4.28 & pandas对接MySQL \\
tqdm & 4.62.3 & 显示循环进度条 \\
\bottomrule
\end{tabular}
\end{center}		

\subsubsection{随机数据生成}
我们小组编写了一系列数据生成器脚本来测试网站。除疫情数据生成器(preprocess.py)放在目录 /database-project/source-code/covid19/ 下,其他生成器统一放在目录 /database-project/source-code/data-generators/ 下。使用生成器需安装5.1.1中列出的第三方库,并修改脚本中的mysql.connector.connect(password)为本地计算机MySQL密码。本小节余下部分详细说明了生成器的使用步骤和方法。

\noindent \textbf{(1)} 建立数据库 \par
使用命令行创建名为\textbf{safe}的数据库。
\begin{lstlisting}[language=SQL]
CREATE DATABASE safe;
\end{lstlisting}

在命令行中进入到 /database-project/source-code/safe/ 目录,运行以下命令。
\begin{lstlisting}[language=bash]
python manage.py makemigrations
python manage.py migrate
\end{lstlisting}

\noindent \textbf{(2)} 生成用户数据 \par 
运行user-information.py。有小概率会碰到Duplicate key错误,原因是Faker生成了重复的姓名,重新运行脚本即可。\par 
class UserGenerator(n, utype) \par 
\noindent 参数:\par 
n: 生成的账号数目 \par 
utype: 0为买方,1为卖方,2为物流公司,3为电商平台,4为物流人员 \par 
\noindent 默认生成200个买方账号,40个卖方账号,10个物流公司账号,5个电商平台账号,50个物流人员账号。

\noindent \textbf{(3)} 生成商品订单数据 \par 
运行order-information.py。\par 
class OrderGenerator(n) \par 
\noindent 参数:\par 
n: 生成的商品订单数目 \par 
\noindent 默认生成400条商品订单记录。

\noindent \textbf{(4)} 生成物流订单数据 \par 
运行delivery-information.py。有小概率会遇到ValueError错误,原因是Faker生成的日期不正确,注释掉model.simulate\_order\_submit()后,重新运行model.simulate\_order\_confirm()即可。\par 
class DeliveryGenerator(n) \par 
\noindent 参数:\par 
n: 生成的物流订单数目 \par 
\noindent 默认生成300条物流订单记录(模拟商家提交订单),并模拟公司确认所有订单。

\noindent \textbf{(5)} 生成公司-员工关系数据 \par
运行company-staff.py。\par 
class CompanyStaffGenerator() \par 
\noindent 生成物流公司的员工列表。

\noindent \textbf{(6)} 生成员工健康数据 \par 
运行person-health.py。\par 
class HealthGenerator() \par 
\noindent 参数:\par 
n: 生成健康信息的天数 \par
\noindent 默认生成所有物流人员2020年11月3日至2021年12月5日间的随机连续十四天健康信息填报数据。

\noindent \textbf{(7)} 生成疫情数据 \par 
运行preprocess.py。将从国家卫生健康委员会的官方网站(\url{http://www.nhc.gov.cn/xcs/yqtb/list_gzbd.shtml})抓取得到的数据进行预处理并导入数据库。covid19article.xlsx中的数据为事先抓取得到的2020年11月3日至2021年12月19日官方疫情通报公文。若需要使用最新数据,请在命令行中进入到 /database-project/source-code/covid19/ 目录下并依次启动爬虫covid19list(抓取文章链接)和covid19article(抓取文章正文)。

\noindent \textbf{注意:}爬虫没有对接ip代理池,运行爬虫有一定概率会被封ip。

\noindent \textbf{(8)} 生成物流订单分配数据 \par 
运行distribution-information.py。\par 
class DistributionGenerator() \par 
\noindent 默认分配所有物流订单,并模拟物流人员确认订单。

\noindent \textbf{(9)} 生成物流位置数据 \par 
运行geographic-information.py。\par 
class GeographicGenerator(n) \par 
\noindent 参数:\par 
n: 在起止坐标之间进行插值的个数 \par
\noindent 默认为每个物流订单生成十二条位置信息记录,坐标限定在中国大陆。

\noindent \textbf{(10)} 生成坐标与城市的对应关系 \par
运行geographic-transformation.py。\par 

\noindent \textbf{(11)} 生成物流公司评分数据 \par
运行company-rate.py。\par 
class CompRateGenerator() \par 
\noindent 默认为所有物流公司生成分项评分数据。

\noindent \textbf{(12)} 生成卖方评分数据 \par
运行seller-rate.py。\par 
class SellerRateGenerator() \par 
\noindent 默认为所有卖方生成分项评分数据。

\subsection{url 设计}
\subsubsection{登录系统} \par
\begin{enumerate}
	\item 主页(游客状态):\url{http://127.0.0.1:8000/safelogistics/}
	\item 注册页面:\url{http://127.0.0.1:8000/safelogistics/sign-up/}
	\item 登录页面:\url{http://127.0.0.1:8000/safelogistics/log-in/} 
\end{enumerate}
\subsubsection{控制面板} \par
根据cookie显示不同类型用户所对应的控制面板界面。\par 
\noindent \textbf{(1)} 买方 \par 
\begin{enumerate}
	\item 主页:\url{http://127.0.0.1:8000/safelogistics/dashboard/} 
	\item 订单总览:\url{http://127.0.0.1:8000/safelogistics/dashboard/information-summary-buyer/}
	\item 订单总览-物流路线分页面:\url{http://127.0.0.1:8000/safelogistics/dashboard/information-summary/visualization/<slug:orderid>/}
	\item 订单总览-风险评估分页面:\url{http://127.0.0.1:8000/safelogistics/dashboard/information-summary/risk/<slug:orderid>/}
	\item 评分查询(查询卖方评分):\url{http://127.0.0.1:8000/safelogistics/dashboard/search-scores-buyer/}
	\item 统计报表:\url{http://127.0.0.1:8000/safelogistics/dashboard/stat-page/}
	\item 设置:\url{http://127.0.0.1:8000/safelogistics/dashboard/profile/}
	\item 设置-修改账户信息分页面:\url{http://127.0.0.1:8000/safelogistics/dashboard/profile/update/}
	\item 设置-切换账户选项:返回登录页面
\end{enumerate}

\noindent \textbf{(2)} 卖方 \par
\begin{enumerate}
	\item 主页:\url{http://127.0.0.1:8000/safelogistics/dashboard/} 
	\item 提交订单:\url{http://127.0.0.1:8000/safelogistics/dashboard/submit-order/}
	\item 订单总览:\url{http://127.0.0.1:8000/safelogistics/dashboard/information-summary-seller/}
	\item 订单总览-物流路线分页面:\url{http://127.0.0.1:8000/safelogistics/dashboard/information-summary/visualization/<slug:orderid>/}
	\item 订单总览-风险评估分页面:\url{http://127.0.0.1:8000/safelogistics/dashboard/information-summary/risk/<slug:orderid>/}
	\item 评分查询(查询物流公司评分):\url{http://127.0.0.1:8000/safelogistics/dashboard/search-scores-seller/}
	\item 统计报表:\url{http://127.0.0.1:8000/safelogistics/dashboard/stat-page/}
	\item 设置:\url{http://127.0.0.1:8000/safelogistics/dashboard/profile/}
	\item 设置-修改账户信息分页面:\url{http://127.0.0.1:8000/safelogistics/dashboard/profile/update/}
	\item 设置-切换账户选项:返回登录页面
\end{enumerate} 

\noindent \textbf{(3)} 物流公司 \par
\begin{enumerate}
	\item 确认订单:\url{http://127.0.0.1:8000/safelogistics/dashboard/confirm-order/}
	\item 确认订单-确认/更新订单分页面:\url{http://127.0.0.1:8000/safelogistics/dashboard/confirm-order/update/<slug:dno>/}
	\item 确认订单-删除选项:删除指定物流订单记录
	\item 订单总览:\url{http://127.0.0.1:8000/safelogistics/dashboard/information-summary-company/}
	\item 订单总览-物流分配分页面:\url{http://127.0.0.1:8000/safelogistics/dashboard/information-summary-company/distribution/<slug:dno>/}
	\item 订单总览-物流路线分页面:\url{http://127.0.0.1:8000/safelogistics/dashboard/information-summary/visualization/<slug:orderid>/}
	\item 订单总览-风险评估分页面:\url{http://127.0.0.1:8000/safelogistics/dashboard/information-summary/risk/<slug:orderid>/}
	\item 员工信息:\url{http://127.0.0.1:8000/safelogistics/dashboard/manage-staffs}
	\item 员工信息-填报历史分页面:\url{http://127.0.0.1:8000/safelogistics/dashboard/manage-staffs/history/<slug:pno>/}
	\item 统计报表:\url{http://127.0.0.1:8000/safelogistics/dashboard/stat-page/}
	\item 设置:\url{http://127.0.0.1:8000/safelogistics/dashboard/profile/}
	\item 设置-修改账户信息分页面:\url{http://127.0.0.1:8000/safelogistics/dashboard/profile/update/}
	\item 设置-切换账户选项:返回登录页面
\end{enumerate}

\noindent \textbf{(4)} 电商平台 \par
\begin{enumerate}
	\item 主页:\url{http://127.0.0.1:8000/safelogistics/dashboard/} 
	\item 订单总览:\url{http://127.0.0.1:8000/safelogistics/dashboard/information-summary-platform/}
	\item 订单总览-物流路线分页面:\url{http://127.0.0.1:8000/safelogistics/dashboard/information-summary/visualization/<slug:orderid>/}
	\item 订单总览-风险评估分页面:\url{http://127.0.0.1:8000/safelogistics/dashboard/information-summary/risk/<slug:orderid>/}
	\item 评分查询(查询卖方评分):\url{http://127.0.0.1:8000/safelogistics/dashboard/search-scores-platform/}
	\item 统计报表:\url{http://127.0.0.1:8000/safelogistics/dashboard/stat-page/}
	\item 设置:\url{http://127.0.0.1:8000/safelogistics/dashboard/profile/}
	\item 设置-修改账户信息分页面:\url{http://127.0.0.1:8000/safelogistics/dashboard/profile/update/}
	\item 设置-切换账户选项:返回登录页面
\end{enumerate}

\noindent \textbf{(5)} 物流人员 \par
\begin{enumerate}
	\item 主页:\url{http://127.0.0.1:8000/safelogistics/dashboard/} 
	\item 确认订单分配:\url{http://127.0.0.1:8000/safelogistics/dashboard/delivery-distribution/}
	\item 每日填报:\url{http://127.0.0.1:8000/safelogistics/dashboard/delivery-health-update/}
	\item 填报历史:\url{http://127.0.0.1:8000/safelogistics/dashboard/delivery-health-view/}
	\item 填报历史-每日填报选项:重定向到每日填报界面
	\item 设置:\url{http://127.0.0.1:8000/safelogistics/dashboard/profile/}
	\item 设置-修改账户信息分页面:\url{http://127.0.0.1:8000/safelogistics/dashboard/profile/update/}
	\item 设置-切换账户选项:返回登录页面	
\end{enumerate}
\subsubsection{用户功能汇总}
\begin{center}
\begin{tabular}{lccccc}
\toprule
\textbf{功能} & \textbf{买家} & \textbf{卖家} & \textbf{物流公司} & \textbf{电商平台} & \textbf{物流人员} \\
\midrule
账户设置 & \checkmark & \checkmark & \checkmark & \checkmark & \checkmark \\
订单总览 & \checkmark & \checkmark & \checkmark & \checkmark & \\
路径可视化 & \checkmark & \checkmark & \checkmark & \checkmark & \\
风险评估 & \checkmark & \checkmark & \checkmark & \checkmark & \\ 
统计报表 & \checkmark & \checkmark & \checkmark & \checkmark & \\
评分查询 & \checkmark & \checkmark &  & \checkmark & \\
填报历史 &  &  & \checkmark  &  & \checkmark \\
提交订单 &  & \checkmark &  &  & \\
确认订单 &  &  & \checkmark  &  & \\
物流分配 &  &  & \checkmark  &  & \\
员工信息 &  &  & \checkmark  &  & \\
确认订单分配 &  &  &  &  & \checkmark \\
每日填报 &  &  &  &  & \checkmark \\
\bottomrule
\end{tabular}
\end{center}


\newpage
\subsection{功能展示}

\subsubsection{登录系统}
\noindent \textbf{(1)} 主页(游客状态)

\begin{center}
	\includegraphics[width = 13.5cm]{home.png}
\end{center}

\noindent \textbf{(2)} 注册页面

\begin{center}
	\includegraphics[width = 13.5cm]{signup.png}
\end{center}

\noindent \textbf{(3)} 登录页面

\begin{center}
	\includegraphics[width = 13.5cm]{login.png}
\end{center}

\subsubsection{控制面板}

\noindent \textbf{(1)} 主页

\begin{center}
	\includegraphics[width = 13.5cm]{buyer-main.png}
\end{center}

\noindent \textbf{(2)} 账户设置

\begin{center}
	\includegraphics[width = 13.5cm]{buyer-setting.png}
\end{center}

\noindent \textbf{(3)} 账户设置:修改账户信息分页面

\begin{center}
	\includegraphics[width = 13.5cm]{modify-profile.png}
\end{center}

\noindent \textbf{(4)} 订单总览

\begin{center}
	\includegraphics[width = 13.5cm]{company-info-sum.png}
\end{center}

\noindent \textbf{(5)} 路径可视化

\begin{center}
	\includegraphics[width = 13.5cm]{buyer-visulization.png}
\end{center}

\noindent \textbf{(6)} 风险评估

\begin{center}
	\includegraphics[width = 13.5cm]{buyer-risk.png}
\end{center}

\noindent \textbf{(7)} 统计报表(公司)

\begin{center}
	\includegraphics[width = 13.5cm]{company-stat.png}
\end{center}

\noindent \textbf{(8)} 统计报表(买方)

\begin{center}
	\includegraphics[width = 13.5cm]{buyer-stat.png}
\end{center}

\newpage
\noindent \textbf{(9)} 统计报表(卖方)

\begin{center}
	\includegraphics[width = 13.5cm]{seller-stat.png}
\end{center}

\newpage
\noindent \textbf{(10)} 统计报表(电商平台)

\begin{center}
	\includegraphics[width = 13.5cm]{platform-stat.png}
\end{center}

\newpage
\noindent \textbf{(11)} 评分查询(查询卖家评分)

\begin{center}
	\includegraphics[width = 13.5cm]{buyer-score-sum.png}
\end{center}

\noindent \textbf{(12)} 评分查询(查询物流公司评分)

\begin{center}
	\includegraphics[width = 13.5cm]{seller-score-sum.png}
\end{center}

\noindent \textbf{(13)} 填报历史(物流公司版)

\begin{center}
	\includegraphics[width = 13.5cm]{company-staff-history.png}
\end{center}

\noindent \textbf{(14)} 填报历史(物流人员版)

\begin{center}
	\includegraphics[width = 13.5cm]{delivery-history.png}
\end{center}

\noindent \textbf{(15)} 提交订单

\begin{center}
	\includegraphics[width = 13.5cm]{seller-submit.png}
\end{center}

\noindent \textbf{(16)} 确认订单

\begin{center}
	\includegraphics[width = 13.5cm]{company-confirm.png}
\end{center}

\noindent \textbf{(17)} 确认订单:确认/更新订单分界面

\begin{center}
	\includegraphics[width = 13.5cm]{company-update.png}
\end{center}

\noindent \textbf{(18)} 物流分配

\begin{center}
	\includegraphics[width = 13.5cm]{company-distribution.png}
\end{center}

\noindent \textbf{(19)} 员工信息

\begin{center}
	\includegraphics[width = 13.5cm]{company-staff.png}
\end{center}

\noindent \textbf{(20)} 确认订单分配

\begin{center}
	\includegraphics[width = 13.5cm]{delivery-confirm.png}
\end{center}

\noindent \textbf{(21)} 每日填报

\begin{center}
	\includegraphics[width = 13.5cm]{delivery-submit.png}
\end{center}

\section{总结}
个人体会及收获,按姓名首字母排序。\par
\vspace{0.3cm}
\noindent\textbf{丁语欣:}\par 
数据库开发设计的这个小项目总体做下来感觉很有趣,也收获满满。做项目的过程让我认识到,课堂学习时举的一些例子都非常经典,经过了简化提炼,但实际生活中,如果我们要进行数据库的设计,需要周全细致的前期工作铺垫以及工作经验辅佐。

以我们小组项目举例,前期按照步骤先完成了ER图、关系模式等设计,但是到了需要构筑数据库,写django的model,实现设想的功能时,又会发现先前设计的模型有这样那样的问题。这个从理论到应用、从纸面到书面的过程还是充满困难和考验的。后期我们就设想的内容,对模式重新进行了梳理和调整,才解决了之前遇到的一些问题。

当然,我们小组所设想的场景仍然是一个非常简化的情况,实际业务中数据之间的关系要复杂得多,在这种情况下前期的调研就格外重要了,我们要从用户、客户以及实际编写代码者等多种角度出发来思考数据库的设计,避免后期搭建过程中对数据关系模式以及依赖关系的频繁改动。
总体而言,这样的项目对我们日后的工作和实践都提供了许多的经验,引导我们进行思考,这个学习思考的过程非常珍贵。

\vspace{0.3cm}
\noindent\textbf{高畅:}\par 
数据库这个project是我在本科生涯中做过的较大的程序设计项目project之一。由于项目涉及到了很多课程之外的内容,比如HTML,CSS和JavaScript等,为了画出更美观的网页,实现更好的用户体验效果,我去学习并更深入的了解了这些语言的语法和应用方式,整个实现我们小组网页和数据库设计的过程中,这些新的知识让我受益匪浅。

同时,也非常感谢我所在的小组所有人都积极参与活动,互相弥补不足,共同努力以设计出让我们满意的程序界面和数据库功能的实现。这次project也让我意识到了,近年来不断改进、不断美化的网页界面的背后是相关程序员的不断努力。为了实现仍和一个小的、看似寻常的改变,程序员可能需要修改或添加数十行乃至上百行代码。虽然这次的数据库project实现情况相较于我们组之前的设想还有很多不足,但我们小组能够齐心协力设计并制作出功能完整的数据库,也彰显着我们在代码上的进步。希望之后如果有闲暇时间,能够进一步完善这个数据库project,使功能更完善,界面更美观。

\vspace{0.3cm}
\noindent\textbf{李舒沛:}\par
实践是检验真理的唯一标准,我觉得这句话在计算机相关的课程上特别适用。理论知识是计算机科学的基础。只有打好了基础,才能最终在技术方面有所成就,而不是满足于会写一两行代码。但是空有理论,没有实践经验在大多数情况下也是不够的(少数情况的例子:做纯理论研究)。对我而言,数据库小组项目是一次很好的实践机会。

老师在课堂上讲解的数据库知识是一套经过时间考验的理论体系,是理解数据库这一计算机重要领域的基础。我觉得系统性理论学习反映了计算机科学静态的一面。能留在课堂和教材上的知识都是经过大浪淘沙后沉淀下来的这门学科的精华。而在项目实践中边做边学则体现了计算机科学动态的一面。目前,计算机科学领域发展迅速,掌握动态学习、快速迭代的技能变得尤为重要。本次项目很好地锻炼了我快速迭代的能力。例如,我通过阅读官方文档和教材,在较短时间内掌握了Git和Django的基本使用方法,进而能够更好地把握项目进度。此外,在整合项目报告的过程中,我也学习到了很多LaTex的有趣功能。本科生涯接近尾声,而这门课是我在四年学习过程中收获最多的课之一。

做小组项目的经过很辛苦,也很快乐。刚开始,我们小组的构想是建立一个完整的疫情风险监控物流系统。做到设计数据库这一步时,我们发现模拟一个真实的物流系统远比想象中要复杂。在老师的耐心指导下,我们定义清楚了小组项目需要实现的核心业务,删除或简化了与数据库不太相关的功能。在项目的编程实践中,我们又根据Django框架的自身特性对部分功能的实现方式进行了调整,如:用django-filter代替复杂的原生SQL查询。我很荣幸能同四位认真负责的同学一起合作,在为期两个多月的项目中不断克服各种困难,见证了我们的项目从几张抽象的草图到一个能够流畅运行的网站。在此,我对她们表示真诚的感谢。

\vspace{0.3cm}
\noindent\textbf{张高阳:}\par
在本学期的数据库课程中,我第一次参与到了小组合作的编程项目中,也是第一次接触网页设计相关内容。作为一个编程新手,这个小组项目对我来说是一个非常大的挑战,因此在项目之初,我不免抱有惴惴不安的心情。在组员的推荐和指导下,我通过各种教程和官方文档主动学习了一些Python、Django以及Git的基础知识,让我对小组项目的具体实施有了更清晰的理解。

在项目实施过程中,我主要负责了有关物流订单、健康信息的确认、提交等相关代码和页面设计。由于我们选择涉及的物流平台数据库涉及多方主体,所以整体设计逻辑会比较复杂,我在项目中边学边做,学习如何根据概念结构设计、逻辑结构设计明确各个模型之间的联系、如何设计网页并实现跳转与链接、如何保证网页的美观性等。这也让我不断克服自己原先对于编程的畏难心理,在看到实际网页呈现时获得了小小的成就感。

其次,我要感谢小组成员们对我的帮助,尤其是组长李舒沛同学在整个项目里统筹安排,对整个小组项目的顺利推进起了非常大的作用。由于我之前的编程经验较少,所以在项目过程中遇到了许多问题,感谢小组成员们能够耐心为我解答。而我们小组前期对数据库的设想比较复杂,这也导致我们在实施过程中遇到很多困难,但是在组员的相互配合下,我们最终还是努力呈现出了相对理想的结果。

总之,这次数据库小组项目让我们把课堂中学到的理论知识更好地消化、吸收,并在实际中进行运用,这让我对数据库设计以及SQL有了更深的认识,也鼓励了我在未来的学习、生活中更多地运用数据库相关知识去解决实际问题。

\vspace{0.3cm}
\noindent\textbf{张秋岑:}\par
在这次小组作业中,我学会了如何设计一个简单的数据库系统。我与组员们一起研究现有系统的构造,确定选题,分析不同用户的功能需求,并将整个流程抽象为一个完整的数据流图。在项目逐步推进的过程中,我们进行了许多次长时间的讨论,不断地对之前已经确定的信息进行修改,我也意识到了在现实工作当中,要设计出一个完整的数据库系统需要考虑许许多多的因素,并且要在一次次讨论中发现问题,并不断优化。比如,在进行用户需求分析时,我需要站在不同用户的视角去思考,在疫情下的物流新场景,每个用户需要哪些信息来作出他们必要的决策。然后再反过来站在数据库系统设计者的视角,探索最高效实现这个功能的方式。最后再回到用户视角,完成界面交互的设计和搭建。

这个站在多个角度思考和设计的过程,让我不仅对数据库有了全面的认知,更是对数据时代信息的流动和意义有了更深刻的理解。

\end{document}